	\thispagestyle{headfootimage}
	\section{Introdução - Produto 3	}
	
	Este capítulo consiste no levantamento e análise do manejo dos resíduos sólidos gerados no município de Monteiro Lobato considerando a caracterização dos resíduos segundo a origem, o volume, as formas de destinação, a disposição final adotada e os respectivos custos associados. Além disso, descreve o acondicionamento, a coleta, o transporte e o transbordo, este quando aplicável, dos seguintes tipos de resíduos:
	
	\begin{itemize}
		\item [primeiro item] Resíduos Domiciliares;
		\item Resíduos de Limpeza Urbana;
		\item Resíduos de Estabelecimentos Comerciais e Prestadores de Serviços;
		\item Resíduos dos Serviços Públicos de Saneamento;
		\item Resíduos Industriais;
		\item Resíduos de Serviços de Saúde;
		\item Resíduos da Construção Civil;
		\item Resíduos Agrossilvipastoris;
		\item Resíduos de Serviços de Transportes; 
		\item Resíduos de Mineração; e
		\item Resíduos de Logística Reversa.
	\end{itemize}
	
	Para a elaboração deste diagnóstico, foi realizada uma reunião preliminar com a \acrlong{smaa} (\acrshort{smaa}), \acrlong{ssm} (\acrshort{ssm}) e vice-prefeito para apresentação do objetivo da elaboração do \gls{pmgirs} de Monteiro Lobato e para compreender as responsabilidades e competências de cada secretaria. Após a reunião, foram encaminhados questionário dirigidos em forma de ofícios para as secretarias com questionamentos detalhados sobre a forma de gestão e gerenciamento dos resíduos sólidos gerados no município. As secretarias realizaram reuniões conjuntas para responder os ofícios e retornaram as respostas por meio eletrônico.
	
	A fim de complementar as informações secundárias apresentadas, foram realizadas visitas de campo acompanhadas pela \gls{smaa} para registros fotográficos do gerenciamento dos resíduos, realização de questionários estruturados para ampliar a participação dos munícipes na elaboração do \gls{pmgirs} e acompanhamento dos serviços de coleta dos resíduos comum e reciclável para determinação da rota.
	
	Os trajetos da coleta e destinação final dos resíduos foram determinados por coordenadas geográficas através do \gls{gps}, modelo Garmin etrex. Esses pontos foram inseridos no software ArcGIS e sobrepostos ao mapa da região.
	
	O questionário estruturado foi aplicado aos estabelecimentos comerciais responsáveis pela geração de óleos lubrificantes e pneus cujos endereços foram disponibilizados pela Prefeitura. O questionário foi realizado para conhecimento da geração e destinação final destes resíduos que devem apresentar sistema de logística reversa de acordo com o artigo 33 da Lei nº 12.305, de 02 de agosto de 2010.
	
	O município de Monteiro Lobato não dispõe de recurso específico destinado para o gerenciamento dos resíduos sólidos e de limpeza urbana. Neste contexto, foi solicitado à \acrshort{ssm}, via ofício, os custos associados para todas atividades do manejo de resíduos sólidos e serviços de limpeza urbana em relação aos salários dos funcionários, manutenção de equipamentos, contratos com empresas terceirizadas e gastos com transporte.
	
	A \acrshort{ssm} repassou um documento em formato .pdf com a relação de funções e atividades de vinte e nove funcionário e a folha de ponto. A fim de estimar qual o custo dos funcionários para a prefeitura em cada atividade de manejo de resíduos sólidos e de serviços de limpeza urbana, o valor total do salário base foi somado aos descontos pagos pela prefeitura (imposto de renda, \acrshort{inss} e demais contribuições) e ao \acrshort{fgts} e este total foi dividido entre os vinte e nove funcionários.
	
	Dependendo do cargo, das atribuições e das horas extras realizadas, a prefeitura ainda arca com uma base variável mensal em relação aos funcionários referente ao pagamento de horas extras e Salário-família. De acordo com a descrição da folha de pagamento disponibilizada, esta base variável é de aproximadamente R\$ 7.202,37 ao mês.
	
	Além disso, o município dispõe de cinco funcionários do Programa Frente de Trabalho, que qualifica profissionalmente e gera renda para cidadãos que se encontram desempregados e em situação de alta vulnerabilidade social, para colaboração em serviços de limpeza urbana, serviços de manutenção e serviços braçais. A estimativa do valor destinado a estes funcionários foi realizada considerando o salário base e o valor da cesta básica que o município disponibiliza para os funcionários.
	
	A partir do diagnóstico de resíduos sólidos é possível desenvolver um sistema de cálculo específico de custos dos serviços de limpeza urbana e de manejo de resíduos sólidos. A partir disso e da elaboração efetiva do \gls{pmgirs} de Monteiro Lobato, o município tem direito a ter acesso a recursos da União destinados para serviços relacionados a limpeza urbana e ao manejo de resíduos sólidos, de acordo com o artigo 18 da \gls{pnrs} (BRASIL, 2010a).
	
	O diagnóstico da situação de resíduos no município de Monteiro Lobato é indispensável para a definição de futuras diretrizes, identificação de gerenciamento inadequados, para o levantamento de ações mitigadoras e preventivas e para a elaboração de prognóstico (BRASIL, 2007a). Além disso, o levantamento do diagnóstico é de suma importância para a elaboração de propostas que visem a utilização racional de recursos ambientais, a redução de desperdícios, a minimização da geração de resíduos sólidos e o manejo ambientalmente adequado e economicamente viável.
	
	\section{Diagnóstico do sistema de limpeza urbana e manejo de resíduos sólidos}
	
	De forma resumida, o artigo 13 da Lei nº 12.305, de 02 de agosto de 2010 classifica os resíduos sólidos como:
	
	\begin{description}
		\item[Resíduos Domiciliares] originários de atividades domésticas em residências urbanas;
		\item[Resíduos de Limpeza Urbana] originários da varrição, limpeza de logradouros e vias públicas e outros serviços de limpeza urbana;
		\item[Resíduos Sólidos Urbanos] englobados nos resíduos domiciliares e nos resíduos de limpeza urbana;
		\item[Resíduos de Estabelecimentos Comerciais e Prestadores de Serviços] gerados nessas atividades, exceto os serviços de limpeza urbana, de saneamento básico, de saúde, de construção civil e de transportes;
		\item[Resíduos de Serviços Públicos de Saneamento Básico] gerados nessas atividades, exceto os resíduos sólidos urbanos;
		\item[Resíduos Industriais] gerados nos processos produtivos e instalações industriais;
		\item[Resíduos de Serviços de Saúde] gerados nos serviços de saúde, conforme definido em regulamento ou em normas estabelecidas pelos órgãos do Sisnama (Sistema Nacional do Meio Ambiente) e do SNVS (Sistema Nacional de Vigilância Sanitária);
		\item[Resíduos da Construção Civil] gerados nas construções, reformas, reparos e demolições de obras de construção civil, incluídos os resultantes da preparação e escavação de terrenos para obras civis;
		\item[Resíduos Agrossilvipastoris] gerados nas atividades agropecuárias e silviculturais, incluídos os relacionados a insumos utilizados nessas atividades;
		\item[Resíduos de Serviços de Transportes] originários de portos, aeroportos, terminais alfandegários, rodoviários e ferroviários e passagens de fronteira;
		\item[Resíduos de Mineração] gerados na atividade de pesquisa, extração ou beneficiamento de minérios;
		\item[Resíduos de Logística Reversa] gerados após o uso pelo consumidor dos materiais provindos de fabricantes, importadores, distribuidores e comerciantes de agrotóxicos e suas embalagens, pilhas e baterias, óleos lubrificantes e suas embalagens, lâmpadas fluorescentes de vapor de sódio e mercúrio e de luz mista, produtos eletroeletrônicos e seus componentes.
		
	\end{description}
	
	\subsection{Resíduos Sólidos Urbanos}
	Os Resíduos Sólidos Urbanos (RSU) compreendem os resíduos domiciliares e os resíduos de limpeza urbana. Os resíduos domiciliares são os resíduos originários de atividades domésticas em residências urbanas. Os resíduos de limpeza urbana englobam os resíduos originários de varrição, de limpeza de logradouros, de vias públicas e de capina e poda (BRASIL, 2010a; BRASIL, 2007a).
	
	Além disso, nos casos em que os resíduos de estabelecimentos comerciais e prestadores de serviços são caracterizados como não perigosos, o poder público pode classificá-los como resíduos equivalentes aos resíduos domiciliares (BRASIL, 2010a).
	
	A geração estimada pela \gls{abrelpe} de RSU no Brasil foi de 78,3 milhões de toneladas em 2016, considerando a soma das projeções de cada região do país. Já a geração per capita apresentou um valor de aproximadamente 1,04 kg por habitante por dia (ABRELPE, 2016).
	
	Em relação a cobertura de coleta, o \gls{ipea} publicou em 2012 uma relação da coleta direta e indireta em área urbana e rural de resíduos sólidos urbanos entre os anos de 2000 e 2008 (tabela 1).