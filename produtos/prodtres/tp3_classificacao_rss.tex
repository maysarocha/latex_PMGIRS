% Table generated by Excel2LaTeX from sheet 'Planilha1'
\begin{table}[htbp]
	\centering
	\caption{Classificação dos RSS quanto aos riscos potenciais de acordo com a Resolução CONAMA 358/2005.}
	\begin{tabular}{|p{35.715em}|}
		\toprule
		\rowcolor[rgb]{ .969,  .588,  .275} Grupo A: Resíduos com a possível presença de agentes biológicos que, por suas características de maior virulência ou concentração, podem apresentar risco de infecção. \\
		\midrule
		\rowcolor[rgb]{ .969,  .588,  .275} A1: Culturas e estoques de microrganismos, entre outros; \\
		\midrule
		\rowcolor[rgb]{ .969,  .588,  .275} A2: Carcaças, peças anatômicas, vísceras, entre outros; \\
		\midrule
		\rowcolor[rgb]{ .969,  .588,  .275} A3: Peças anatômicas (membros) do ser humano, entre outros; \\
		\midrule
		\rowcolor[rgb]{ .969,  .588,  .275} A4: Kits de linhas arteriais, endovenosas e deslizadores, quando descartado, e outros; \\
		\midrule
		\rowcolor[rgb]{ .969,  .588,  .275} A5: Órgãos, tecidos, fluidos orgânicos, entre outros. \\
		\midrule
		\rowcolor[rgb]{ .984,  .831,  .706} Grupo B: Resíduos contendo substâncias químicas que podem apresentar risco à saúde pública ou ao meio ambiente, dependendo de suas características de inflamabilidade, corrosividade, reatividade e toxicidade, entre outros; \\
		\midrule
		\rowcolor[rgb]{ .969,  .588,  .275} Grupo C: Quaisquer materiais resultantes de atividades humanas que contenham radionuclídeos em quantidades superiores aos limites de eliminação especificados nas normas da Comissão Nacional de Energia Nuclear – CNEN e para os quais a reutilização é imprópria ou não prevista. \\
		\midrule
		\rowcolor[rgb]{ .984,  .831,  .706} Grupo D: Resíduos que não apresentem risco biológico, químico ou radiológico à saúde ou ao meio ambiente, podendo ser equiparados aos resíduos domiciliares. \\
		\midrule
		\rowcolor[rgb]{ .969,  .588,  .275} Grupo E: Materiais perfurocortantes ou escarificantes, entre outros; \\
		\bottomrule
	\end{tabular}%
		\label{tab:classificacao_rss}
\end{table}%

%

