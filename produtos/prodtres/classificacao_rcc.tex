% Table generated by Excel2LaTeX from sheet 'classificacao_rcc'
\begin{table}[htbp]
  \centering
  \caption{Classificação dos Resíduos de Construção Civil.}
    \begin{tabular}{c|p{8cm}}
    \rowcolor[rgb]{ .984,  .831,  .706} \textbf{I - Classe A} & São os resíduos reutilizáveis ou recicláveis como agregados, tais como:  \\
    \rowcolor[rgb]{ .984,  .831,  .706}       & a) de construção, demolição, reformas e reparos de pavimentação e de outras obras de infraestrutura, inclusive solos provenientes de terraplanagem; \\
    \rowcolor[rgb]{ .984,  .831,  .706}       & b) de construção, demolição, reformas e reparos de edificações: componentes cerâmicos (tijolos, blocos, telhas, placas de revestimento etc.), argamassa e concreto;  \\
    \rowcolor[rgb]{ .984,  .831,  .706}       & c) de processo de fabricação e/ou demolição de peças pré-moldadas em concreto (blocos, tubos, meio-fios etc.) produzidas nos canteiros de obras;  \\
    \rowcolor[rgb]{ .992,  .914,  .851} \textbf{II - Classe B} & São os resíduos recicláveis para outras destinações, tais como: plásticos, papel/papelão, metais, vidros, madeiras e outros;  \\
    \rowcolor[rgb]{ .984,  .831,  .706} \textbf{III - Classe C} & São os resíduos para os quais não foram desenvolvidas tecnologias ou aplicações economicamente viáveis que permitam a sua reciclagem/recuperação, tais como os produtos oriundos do gesso;  \\
    \rowcolor[rgb]{ .992,  .914,  .851} \textbf{IV - Classe D} & São os resíduos perigosos oriundos do processo de construção, tais como: tintas, solventes, óleos e outros, ou aqueles contaminados oriundos de demolições, reformas e reparos de clínicas radiológicas, instalações industriais e outros, bem como telhas e demais objetos e materiais que contenham amianto ou outros produtos nocivos à saúde. \\
    \end{tabular}%
  \label{tab:classificacao_rcc}%
  \legend{Fonte: Adaptado de CONAMA, 2002b.}
\end{table}%
